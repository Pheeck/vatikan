\documentclass{article}

\usepackage[czech]{babel}
\usepackage[utf8]{inputenc}
\usepackage{geometry}
\usepackage{csquotes}
\usepackage{hyperref}
\usepackage{graphicx}
\usepackage{float}
\usepackage{fdsymbol}

\graphicspath{{./}}

\title{Programátorská dokumentace karetní počítačové hry Vatikán}

\begin{document}

\maketitle

\section*{Co to je}

Tento dokument je programátorská dokumentace počítačové verze karetní hry
Vatikán. Existuje více variant této hry. Naimplementoval jsem variantu, která
se hraje se dvěma balíčky žolíkových karet ovšem bez žolíků a hráči se střídají
ve vykládání karet. Naimplementoval jsem pouze hru dvou hráčů. Jeden nebo
oba hráči mohou být ovládáni jednoduchou umělou inteligencí.

Program jsem vyvíjel s Pythonem 3.12.3 a Pygame 2.5.2 pro Linux a Windows. Pro
vytvoření samostatné binárky pro Windows jsem použil knihovnu pyinstaller a
postup popsaný v tomto Stack Overflow příspěvku
\url{https://stackoverflow.com/a/54926684}. Použil jsem volně dostupné assety
uživatele yaomon stránky itch.io -- \url{https://yaomon.itch.io/playing-cards}.

\subsection*{Pravidla}

Zde je přehled pravidel hry.
\begin{itemize}
	\item Cílem hry je vyložit všechny karty z ruky
	\item Ve svém tahu může hráč vykládat karty ze svojí ruky na stůl a
		přemisťovat karty na stole. Nemůže brát karty ze stolu. V
		počítačové implementaci hry \emph{je} možné brát karty ze
		stolu, avšak pouze ty, které byly stejné kolo vyloženy.
		Efektivně to znamená, že hráč v průběhu tahu může vrátit
		rozhodnutí vyložit kartu, což mu umožňuje lépe tah rozmýšlet.
	\item Karty na stole mají být uspořádány do \emph{stacků} (hromádek),
		kde každá hromádka musí být buď \emph{flush} nebo
		\emph{triplet}, viz sekce Pojmy. Počítačová implementace
		splnění tohoto pravidla vyžaduje na konci tahu.
	\item Pokud hráč za svůj tah nevyložil žádnou kartu ze své ruky, musí
		si před ukončením tahu jednu kartu líznout z dobíracího
		balíčku.
\end{itemize}

\section*{Pojmy}

Zde vysvětluji některé pojmy používané v tomto dokumentu a v komentářích v
kódu.

\begin{itemize}
	\item Frozen karta -- Karta, která nezačínala tento tah v ruce. Nelze
		ji vzít do ruky. Pokud všechny karty na stole jsou frozen, je
		stůl považován za frozen. Frozen stůl znamená dolíznutí karty
		při ukončení tahu.
	\item Stack (hromádka) -- Skupina (může být prázdná) karet (třída
		\texttt{Card}). Stack může být reprezentován třídou
		\texttt{Stack} nebo jednodušeji pouze tuplem (v případě modulu
		\texttt{ai}).
	\item Validní stack -- Buď prázdný stack nebo stack obsahující flush
		nebo triplet.
	\item Cyclic order -- Uspořádání karet dle hodností takové, že po
		nejvyšší kartě následuje nejnižší. Např. posloupnosti \texttt{4
		5 6}, \texttt{10 J Q K} a \texttt{Q K A 2 3} mají cyclic order.
	\item Flush (postupka) -- Skupina alespoň 3 karet jedné barvy, které
		lze uspořádat do cyclic orderu. Žádná z karet se nesmí
		opakovat. Např. \texttt{A 2 3} je flush, \texttt{4 5} a
		\texttt{2 3 9 10} nikoliv.
	\item Triplet -- Skupina alespoň 3 karet stejné hodnosti. Žádná z karet
		se nesmí opakovat. Pozorujme, že tedy triplet může mít jedině 3
		nebo 4 karty. Např. \texttt{$\clubsuit$Q $\diamondsuit$Q
		$\heartsuit$Q}.
	\item Widget -- Objekt třídy z modulu \texttt{widgets}. Reprezentuje
		nějakou část uživatelského rozhraní a zároveň logiky hry. Více
		viz sekce Třídy v modulu \texttt{widgets}.
	\item Missing marker -- Značka (reprezentována hodnotou \texttt{None})
		signalizující, že skupina karet téměř tvoří \emph{flush},
		avšak na místě značky chybí karta. Missing markerů ve skupině
		karet může být více. Koncept missing markerů je v programu
		přítomen hlavně z estetických důvodů. Slouží k tomu, aby
		uživatelské rozhraní mohlo dát lépe najevo, že hromádka je
		téměř validní. Algoritmus pro testování validity \emph{flushe}
		běžně vrací tuply obsahující missing markery. Např. \texttt{K A
		2 None 3 4}.
\end{itemize}

\section*{Jak to funguje}

\subsection*{O co se stará která třída / modul}

\subsubsection*{Třída \texttt{Card}}

Všechny karty jsou reprezentovány třídou \texttt{Card}. Každá karta je
vytvořena za běh programu pouze při incializaci třídy \texttt{Game}. Poté již
objekty třídy \texttt{Card} nevznikají ani nezanikají. Kartě je možné nastavit
příznak \emph{frozen}. To se u každé karty stane nejvýše jednou za hru a
příznak už není možné odebrat.

\subsubsection*{Třída \texttt{Game}}

Třída \texttt{Game} je hlavní třídou programu. Instanciuje se za jeho běh pouze
jednou. Vlastní veškeré \emph{widgety} a společně s nimi reprezentuje stav hry.
Její metoda \texttt{run} obsahuje hlavní cyklus hry. Veškeré změny stavu hry se
dějí skrz \enquote{API} tvořené metodami
\begin{verbatim}
try_end_turn
try_take_card_from_stack
try_take_card_from_hand
try_put_card_onto_stack
try_put_card_into_hand
find_stack_containing_cards
get_random_empty_stack
get_state_copy
\end{verbatim}
Tyto metody využívá jak metoda \texttt{Game.process\_mouse\_click}, starající
se o zpracování příkazů od uživatele, tak funkce \texttt{ai.apply\_moves} z
modulu \texttt{ai} starající se o zpracování příkazů umělé inteligence.

\subsubsection*{Třídy v modulu \texttt{widgets}}

Logika hry je implementována částečně ve třídě \texttt{Game} a částečně ve
\emph{widgetech}. Ty jsou zároveň schopné vykreslit se na obrazovku a
odpovídat, jestli na ně bylo kliknuto.

Třída \texttt{Stack} reprezentuje stack karet. Dokáže odpovídat na otázku,
jestli je validní. Udržuje si seznam karet dvakrát -- jednou bez \emph{missing
markerů} a jednou s nimi. Třída \texttt{Hand} reprezentuje karty v ruce hráče.
Ve hře je vždy instanciovaná dvakrát -- pro každého z hráčů. Widgety třídy
\texttt{PickUpArea} zobrazují, která karta je právě zvolena. Přesouvání karty
probíhá následovně. Karta je nejprve zvolena, čímž opustí stack/ruku. V tu
chvíli objekt karty vlastní pickup widget. Poté je zvolená karta umístěna do
ruky nebo do nějakého stacku. Tyto akce probíhají pomocí API třídy
\texttt{Game}. \texttt{PickUpArea} je také vždy instanciována dvakrát. Třída
\texttt{Deck} reprezentuje dobírací balíček. Vlastní karty, které ještě nejsou
ve hře. Zobrazuje počet karet v dobíracím balíčku. Třída \texttt{EndTurnButton}
umožňuje uživateli ukončit kolo a zobrazuje stav stolu. Napovídá, jestli
všechny stacky jsou validní a tedy je možné kolo ukončit a jestli na stole leží
pouze \emph{frozen} karty, tedy ukončení kola znamená dolíznutí karty.

\subsubsection*{Modul \texttt{ai}}

Modul \texttt{ai} obsahuje tři hlavní funkce \texttt{generate\_moves},
\texttt{print\_moves} a \texttt{apply\_moves}. Funkce \\ \texttt{generate\_moves}
si zkopíruje stav hry, použije algoritmus, který popisuji v sekci Umělá
inteligence a vydá informace o moves (přesunech karet), které umělá inteligence
vyhodnotila jako nejlepší. Funkce se dívá pouze na možnosti v rámci jednoho
tahu. Nijak nebere v úvahu budoucí tahy svoje ani protivníkovy. Funkce
\texttt{print\_moves} vypisuje moves do terminálu v lidsky čitelné podobě.
Funkce \texttt{apply\_moves} za pomocí API třídy \texttt{Game} moves aplikuje.

\subsubsection*{Třída \texttt{Menu}}

Třída \texttt{Menu} reprezentuje výběr z herních režimů, který se ukáže
uživateli po spuštění programu. Instanciuje se pouze jedenkrát za běh programu.
Má vlastní hlavní cyklus. Když uživatel zvolí herní režim, hlavní cyklus třídy
skončí a incializuje se třída \texttt{Game} s herním režimem jako jedním z
argumentů.

\subsection*{Algoritmy}

\subsubsection*{Je toto flush?}

Pojďme teď rozebrat algoritmus vyhodnocující, jestli daná skupina karet tvoří
\emph{flush}. Protože rozpoznat validní triplet je snadné, tento algoritmus je
jedinou netriviální částí rozpoznávání validního \emph{stacku}.

Algoritmus nejprve zkontroluje, že skupina obsahuje alespoň 3 karty, všechny
mají stejnou barvu a žádná není ve stacku dvakrát (kouká, jestli skupina
obsahuje unikátní ranky). Zásadní myšlenka algoritmu je, že dovolíme přidávat
\emph{missing markery} na místa karet, která ve \emph{flushi} chybí. Pokud jsme
splnili iniciální podmínky, je už v tuto chvíli jisté, že \emph{flush} (ač
možná s \emph{missing markery}) dokážeme sestrojit. Nyní algoritmus buď vydá
skupinu bez \emph{missing markerů} a my víme, že máme \emph{flush} a nebo vydá
setříděnou skupinu s \emph{missing markery}, což znamená, že vstup neobsahoval
\emph{flush} a zároveň signalizuje, jaké karty je třeba přidat.

Ovšem my budeme od algoritmu požadovat ještě jednu dodatečnou vlastnost --
chceme, aby algoritmus umístil nejmenší možný počet \emph{missing markerů}.
Seřadíme si karty podle ranků. Některé ranky typicky nebudou přítomné. Najdeme
nejdelší souvislý úsek chybějících ranků -- největší mezeru. Zrotujeme
posloupnost karet tak, aby tato mezera byla \enquote{na vnějšku} -- aby byla
mezi posledním a prvním prvkem. Nyní iterujeme od ranku první karty
posloupnosti po rank poslední karty z posloupnosti. Pokud rank v posloupnosti
je, vydáme odpovídající kartu. Pokud ne, vydáme missing marker. Tudíž vyplníme
všechny mezery kromě té největší. Nejvýše jedna mezera již znamená, že máme
\emph{flush}. Protože nevyplněná zůstala největší mezera, použili jsme nejmenší
možný počet \emph{missing markerů}.

\subsubsection*{Umělá inteligence}

Nyní rozeberme, jak umělá inteligence určuje, které tahy provést. Můj původní
plán byl prohledávat stavový prostor a ze stavů, kde všechny stacky jsou
validní vybrat ten, který znamená nejvíce vyložených karet. Ovšem prvotní
pokusy odhalily, že vzhledem k počtu možných přesunutí karet je prohledávání
stavového prostoru výpočetně extrémně náročné a tedy není praktickým řešením.

Zvolil jsem tedy alternativní přístup, kde neuvažuji všechny možnosti, jak
vyložit karty na stůl. Uvažuji pouze ty možnosti, o kterých jsem empiricky
zjistil, že je hráči hry dělají nejčastěji. Tedy implementace umělé inteligence
zkouší tyto moves
\begin{enumerate}
	\item Vytvoř nový stack o třech kartách
		\begin{enumerate}
			\item Použij výhradně karty z ruky
			\item Použij jednu krajovou kartu z některého velkého
				stacku. Velký stack má alespoň 4 karty. Např.
				ze stacku \texttt{$\heartsuit$4 $\heartsuit$5
				$\heartsuit$6 $\heartsuit$7} vezmi kartu
				\texttt{$\heartsuit$7}.
			\item Použij dvě krajové karty z některých velkých
				stacků
		\end{enumerate}
	\item Přilož kartu z ruky k existujícímu stacku
\end{enumerate}
Pro jednoduchost implementace jsem možnost 1c dále omezil tím, že neuvažuji
odebrání dvou karet z jednoho stacku (což by u stacku o alespoň pěti kartách
bylo možné se zachováním jeho validity). Algoritmus nejprve zkouší aplikovat
moves typu 1a dokud se nepřesvědčí, že žádný další aplikovat nelze, poté 1b, 1c
a nakonec 2.

Tento přístup se osvědčil. Vzhledem k jeho jednoduchosti dává překvapivě
kompetentního, ač docela slabého, automatického protivníka. Navíc výsledný
algoritmus je polynomiální a běží dost rychle na to, aby prodleva
\enquote{přemýšlení} umělé inteligence byla prakticky nepostřehnutelná.

Algoritmus oproti lidskému hráči selhává v situacích, kdy už v ruce má velice
málo karet -- jednu nebo dvě. Tehdy typicky standardní moves nejsou k dispozici
a umělá inteligence tedy neví, co dělat. Lidský hráč v tu chvíli rozšíří
množinu moves, nad kterou uvažuje a je ochotný strávit delší čas přemýšlením
nad tím, jak přeskládat hromádky na stole, aby mohl své poslední karty vynést.
Potencionální zlepšení umělé inteligence by mohlo být přece jen se u konce hry
obrátit na nějaké vhodně omezené prohledávání stavového prostoru -- mimikovat
chování člověka.

\end{document}

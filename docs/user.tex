\documentclass{article}

\usepackage[czech]{babel}
\usepackage[utf8]{inputenc}
\usepackage{geometry}
\usepackage{csquotes}
\usepackage{hyperref}
\usepackage{graphicx}
\usepackage{float}

\graphicspath{{./}}

\title{Manuál ke karetní počítačové hře Vatikán}

\begin{document}

\maketitle

\section*{Co to je}

Tento dokument je manuál k počítačové verzi karetní hry Vatikán. Hra (popř.
její varianty) je také známá pod názvy Autobus nebo Amerika. Počítačová hra by
měla být hratelná na MS Windows nebo na vesměs libovolné distribuci Linuxu.

\section*{Jak to spustit}

\subsection*{Windows}

Je-li váš operační systém Windows, lze hru spustit jednoduše poklepáním na
spustitelný soubor \texttt{vatikan.exe}.

Následně by se mělo otevřít okno se hrou zobrazující hlavní menu, nápis
\emph{Karetni hra VATIKAN} a tři tlačítka odpovídající herním režimům.

\subsection*{Linux a pokročilí uživatelé Windows}

Zde je alternativní způsob, jak hru spustit. Tento způsob funguje pro Linux i
Windows. Je třeba mít nainstalovaný interpret Python~3 a knihovnu Pygame 2.
Během vývoje jsem používal Python 3.12.3 a Pygame 2.5.2.

Ve složce se hrou spusťte příkaz

\begin{verbatim}
	python3 __main__.py
\end{verbatim}

\section*{Jak to hrát}

\subsection*{Po spuštění}

Hra se ovládá výhradně myší. Po spuštění hry se dostanete do hlavního menu, kde
dostanete na výběr ze tří herních režimů
\begin{itemize}
	\item Hráč proti hráči -- Střídají se tahy dvou hráčů. Lze hrát sám
		proti sobě. Lze hrát u jednoho počítače ve dvou. Karty obou
		hráčů jsou stále odkryté. Co se týče herní plochy, hráč 1 je
		dole, hráč 2 je nahoře.
	\item Hráč proti AI -- Střídá se hráč a počítač. Karty počítače jsou
		zakryté. Počítač předstírá, že karty hráče nezná. Hráč je dole,
		počítač je nahoře.
	\item AI proti AI -- Počítač hraje sám proti sobě. V tomto režimu nelze
		s herní plochou interagovat. Režim primárně určený k testování
		hry.
\end{itemize}

\begin{figure}[H]
	\begin{center}
		\includegraphics[height=150px]{main-menu.png}
	\end{center}
\end{figure}

Herní režim zvolíte klepnutím myší na odpovídající tlačítko. Zbytek manuálu
bude předpokládat, že jste zvolili režim \enquote{Hráč proti AI}.

\subsection*{Přesouvání karet}

Nyní byste měli mít před sebou herní plochu a karty ve svojí ruce. Kliknutím na
kartu ji zvolíte. Zvolená karta se vždy nachází v levém dolním rohu obrazovky.
Pokud máte zvolenou nějakou kartu, můžete ji kliknutím na ruku vrátit zpět do
ruky nebo ji můžete klepnutím na některé z míst na stole vyložit na stůl.
Založíte tak novou hromádku. Další karty můžete vykládat na volná místa nebo
přikládat na existující hromádky. Lze zvolit nejen karty z ruky, ale také
karty, které již leží na stole v některé z hromádek.

\begin{figure}[H]
	\begin{center}
		\includegraphics[height=150px]{moving-cards.png}
	\end{center}
\end{figure}

\subsection*{Cíl hry a správné hromádky}

Cílem hry je zbavit se všech karet v ruce a ukončit tah. Avšak tah lze ukončit
pouze pokud všechny hromádky na stole jsou \enquote{správné}. Správná hromádka
je buď \enquote{postupka}, kde navazují hodnosti a karty mají stejnou barvu,
nebo \enquote{trojice/čtveřice}, kde hodnost je stejná a karty mají různé
barvy. Pořadí karet v postupce je \texttt{2 3 4 5 6 7 8 9 10 J K Q A} a
postupka je chápána cyklicky, tedy po \texttt{A} následuje \texttt{2}. Každá
správná hromádka má alespoň 3 karty a neobsahuje jednu kartu dvakarát. Když
hromádka není správná, zbarví se dočervena.

Tah můžete ukončit tak, že klepnete na tlačítko nacházející se vpravo uprostřed
okna pod obrázkem dobíracího balíčku. Pokud některá hromádka není správná,
tlačítko nebude funkční. Pokud jste za tah nevynesli žádnou kartu z ruky,
s koncem tahu si automaticky doberete kartu z dobíracího balíčku.

\begin{figure}[H]
	\begin{center}
		\includegraphics[height=150px]{ending-turn.png}
	\end{center}
\end{figure}

\subsection*{Zmražené karty}

Karty z minulých tahů je zakázáno brát do ruky. Není možné ukončit tah, pokud
je zvolena některá z těchto karet.

\section*{Poděkování/Acknowledgements}

Rád bych poděkoval uživateli \texttt{yaomon} stránky \texttt{itch.io}, jehož
obrázky jsem použil jako obrázky karet v této hře.

I'd like to give credit to user \texttt{yaomon} of the website \texttt{itch.io}
whose artwork I used as assets in this game.

\url{https://yaomon.itch.io/playing-cards}

\end{document}
